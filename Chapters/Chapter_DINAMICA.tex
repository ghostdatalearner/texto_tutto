% Chapter Template

\chapter{Modelado dinámico} % Main chapter title

\label{DINAMICA} % Change X to a consecutive number; for referencing this chapter elsewhere, use \ref{ChapterX}

%----------------------------------------------------------------------------------------
%	SECTION 1
%----------------------------------------------------------------------------------------

\section{Dinámica de las comunidades mutualistas}

Lorem ipsum dolor sit amet, consectetur adipiscing elit. Aliquam ultricies lacinia euismod. Nam tempus risus in dolor rhoncus in interdum enim tincidunt. Donec vel nunc neque. In condimentum ullamcorper quam non consequat. Fusce sagittis tempor feugiat. Fusce magna erat, molestie eu convallis ut, tempus sed arcu. Quisque molestie, ante a tincidunt ullamcorper, sapien enim dignissim lacus, in semper nibh erat lobortis purus. Integer dapibus ligula ac risus convallis pellentesque .

%-----------------------------------
%	SUBSECTION 1
%-----------------------------------
\subsection{Modelos de población}

El uso de modelos cuantitativos en el estudio de la dinámica de poblaciones fue una de las primeras aplicaciones de las matemáticas en el campo de la biología, con antecedentes tan remotos como Fibonacci y Malthus. Todo modelo supone una descripción simplificada del fenómeno que se quiere estudiar y las formulaciones clásicas, como la de crecimiento de Verhulst o la de interacción presa-depredador de Lotka-Volterra  resultaban muy atractivas por su sencillez, aunque limitadas a la hora de aplicarlas a escenarios reales. Los modelos se fueron refinando, pero el paradigma se mantuvo hasta finales del siglo XX

\section{Modelo con capacidad de carga constante}

Nunc posuere quam at lectus tristique eu ultrices augue venenatis. Vestibulum ante ipsum primis in faucibus orci luctus et ultrices posuere cubilia Curae; Aliquam erat volutpat. Vivamus sodales tortor eget quam adipiscing in vulputate ante ullamcorper. Sed eros ante, lacinia et sollicitudin et, aliquam sit amet augue. In hac habitasse platea dictumst.

Probando fórmulas. Como dice la fórmula \ref{myeq1}...

\begin{align}
\displaystyle &\frac{dN}{dt}=N\, \left(a-b \,P\right), \nonumber\\
\displaystyle &\frac{dP}{dt}=P\, \left(c\, N-d\right) , 
\label{myeq1}
\end{align}

Otra fórmula más. Como se demuestra en \ref{eq:reffs_2especies}...
\begin{align}
A = & \, r_{1}+ b_{12}\, {N_2^a}^0 - (\alpha_{1}+ c_{1} \, b_{12}\, {N_{2}^a}^0) \, {N_1^p}^0 , \nonumber\\
-B = &\, r_{2} + b_{21} \, {N_{1}^p}^0-(\alpha_{2}+ c_{2}\,  b_{21}\, {N_{1}^p}^0)\,  {N_{2}^a}^0 .
\label{eq:reffs_2especies}
\end{align}

\subsection{Análisis de estabilidad}

Nunc posuere quam at lectus tristique eu ultrices augue venenatis. Vestibulum ante ipsum primis in faucibus orci luctus et ultrices posuere cubilia Curae; Aliquam erat volutpat. Vivamus sodales tortor eget quam adipiscing in vulputate ante ullamcorper. Sed eros ante, lacinia et sollicitudin et, aliquam sit amet augue. In hac habitasse platea dictumst.

\section{Modelo con saturación del beneficio}

Nunc posuere quam at lectus tristique eu ultrices augue venenatis. Vestibulum ante ipsum primis in faucibus orci luctus et ultrices posuere cubilia Curae; Aliquam erat volutpat. Vivamus sodales tortor eget quam adipiscing in vulputate ante ullamcorper. Sed eros ante, lacinia et sollicitudin et, aliquam sit amet augue. In hac habitasse platea dictumst.

\subsection{Análisis de estabilidad}

Nunc posuere quam at lectus tristique eu ultrices augue venenatis. Vestibulum ante ipsum primis in faucibus orci luctus et ultrices posuere cubilia Curae; Aliquam erat volutpat. Vivamus sodales tortor eget quam adipiscing in vulputate ante ullamcorper. Sed eros ante, lacinia et sollicitudin et, aliquam sit amet augue. In hac habitasse platea dictumst.

\section{Resultados}

Nunc posuere quam at lectus tristique eu ultrices augue venenatis. Vestibulum ante ipsum primis in faucibus orci luctus et ultrices posuere cubilia Curae; Aliquam erat volutpat. Vivamus sodales tortor eget quam adipiscing in vulputate ante ullamcorper. Sed eros ante, lacinia et sollicitudin et, aliquam sit amet augue. In hac habitasse platea dictumst.

\section{Conclusiones}

Nunc posuere quam at lectus tristique eu ultrices augue venenatis. Vestibulum ante ipsum primis in faucibus orci luctus et ultrices posuere cubilia Curae; Aliquam erat volutpat. Vivamus sodales tortor eget quam adipiscing in vulputate ante ullamcorper. Sed eros ante, lacinia et sollicitudin et, aliquam sit amet augue. In hac habitasse platea dictumst.