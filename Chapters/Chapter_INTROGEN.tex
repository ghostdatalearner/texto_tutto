% Chapter Template

\chapter{Introducción general} % Main chapter title

\label{INTROGEN} % Change X to a consecutive number; for referencing this chapter elsewhere, use \ref{ChapterX}

%----------------------------------------------------------------------------------------
%	SECTION 1
%----------------------------------------------------------------------------------------

La Ecología estudia las interacciones entre especies y entre ellas  y su entorno. Como rama de la Biología, tiene un desarrollado sentido de la clasificación y denomina a las de la primera clase \textit{bióticas} y \textit{abióticas} a las de la segunda. La gama de las bióticas es extensa, pero también se reduce a unas pocas categorías. En la \textit{depredación} una especie se beneficia y otra resulta perjudicada, como en el \textit{parasitismo}, mientras que en la \textit{competencia} los individuos, ya sean o no de la misma especie, disputan un mismo recurso. Estos tres tipos de relaciones se caracterizan por la realimentación negativa, resultan perjudiciales para una de las partes y beneficiosas para la otra. 

Si una de las especies obtiene beneficio pero para la otra es neutral se trata de \textit{comensalismo}. Finalmente, si la relación es positiva para ambas especies, es \textit{mutualismo}. En Ecología se contemplan estas relaciones desde un punto de vista de funcional y mecanicista \cite{rockwood2006introduction}, atendiendo sobre todo a los flujos de intercambio de materia, energía o servicios.

El conjunto de interacciones crea sistemas de gran complejidad. La dinámica de las poblaciones, puede describirse utilizando modelos matemáticos. Además, en las últimas dos décadas, la ciencia de redes ha contribuido al conocimiento de las comunidades ecológicas aplicando sus propias técnicas de análisis. Estas descripciones, fundamentadas en modelos y propiedades topológicas, son fenomenológicas \cite{van2010ethical}.

Los distintos enfoques metodológicos suponen a veces dificultades de comunicación, pero se enriquecen. En esta tesis se plantean contribuciones al estudio del \textit{mutualismo} desde el modelado matemático y la ciencia de redes, intentando no perder de vista su significado biológico.

\section{El mutualismo en ecología}

El término \textit{mutualismo} tuvo su origen en economía política a principios del XIX, relacionado con distintas concepciones utópicas. Fue el filósofo Pierre Joseph Proudhon el que hizo del mutualismo el eje de su teoría social y económica.

\enquote{\itshape [Mutualismo es] un sistema de equilibrio entre fuerzas libres, en el cual está cada una segura de gozar de los mismos derechos bajo la condición de llenar los mismos deberes, y 
de obtener las mismas ventajas a cambio de los mismos servicios} \cite{proudhon1868capacite}.

La idea de beneficio compartido fue trasladada al campo de la biología por el parasitólogo belga Pierre-Joseph van Beneden \cite{boucher1982ecology}, que escribió:

\enquote{\itshape Al lado [de los parásitos y comensales] hay otros que se prestan mutuamente servicios [..]. Creemos que es más justo llamarles Mutualistas} \cite{van1878commensaux}.

El mutualismo puede adoptar varias formas e intensidades. La característica que lo diferencia del resto de relaciones ecológicas es la cooperación entre especies mediante que intercambian servicios o bienes \cite{bronstein2001exploitation}.

Una distinción se basa en la importancia vital para los actores. En el mutualismo obligatorio ambas especies requieren del concurso de la otra para subsistir. Se suelen citar los ejemplos de la anénoma y el pez payaso o de la yuca y sus polillas \textit(Prodixidae), aunque hay dudas de que sean absolutamente obligatorios \cite{briand1982phylogenetic, addicott1995cheating}. Está muy asociado a una gran especialización y coevolución de los mutualista. En el facultativo, la relación no tiene ese carácter esencial. Es el más común en las comunidades de plantas y polinizadores \cite{geib2012tracing}


%-----------------------------------
%	SUBSECTION 1
%-----------------------------------
\subsection{Tipos de mutualismo}

Nunc posuere quam at lectus tristique eu ultrices augue venenatis. Vestibulum ante ipsum primis in faucibus orci luctus et ultrices posuere cubilia Curae; Aliquam erat volutpat. Vivamus sodales tortor eget quam adipiscing in vulputate ante ullamcorper. Sed eros ante, lacinia et sollicitudin et, aliquam sit amet augue. In hac habitasse platea dictumst.

%-----------------------------------
%	SUBSECTION 2
%-----------------------------------

\subsection{Historia de los estudios sobre mutualismo}

Morbi rutrum odio eget arcu adipiscing sodales. Aenean et purus a est pulvinar pellentesque. Cras in elit neque, quis varius elit. Phasellus fringilla, nibh eu tempus venenatis, dolor elit posuere quam, quis adipiscing urna leo nec orci. Sed nec nulla auctor odio aliquet consequat. Ut nec nulla in ante ullamcorper aliquam at sed dolor. Phasellus fermentum magna in augue gravida cursus. Cras sed pretium lorem. Pellentesque eget ornare odio. Proin accumsan, massa viverra cursus pharetra, ipsum nisi lobortis velit, a malesuada dolor lorem eu neque.

%----------------------------------------------------------------------------------------
%	SECTION 2
%----------------------------------------------------------------------------------------

\section{Redes en ecología}

Sed ullamcorper quam eu nisl interdum at interdum enim egestas. Aliquam placerat justo sed lectus lobortis ut porta nisl porttitor. Vestibulum mi dolor, lacinia molestie gravida at, tempus vitae ligula. Donec eget quam sapien, in viverra eros. Donec pellentesque justo a massa fringilla non vestibulum metus vestibulum. Vestibulum in orci quis felis tempor lacinia. Vivamus ornare ultrices facilisis. Ut hendrerit volutpat vulputate. Morbi condimentum venenatis augue, id porta ipsum vulputate in. Curabitur luctus tempus justo. Vestibulum risus lectus, adipiscing nec condimentum quis, condimentum nec nisl. Aliquam dictum sagittis velit sed iaculis. Morbi tristique augue sit amet nulla pulvinar id facilisis ligula mollis. Nam elit libero, tincidunt ut aliquam at, molestie in quam. Aenean rhoncus vehicula hendrerit.

\subsection{Redes mutualistas}
Morbi rutrum odio eget arcu adipiscing sodales. Aenean et purus a est pulvinar pellentesque. Cras in elit neque, quis varius elit. Phasellus fringilla, nibh eu tempus venenatis, dolor elit posuere quam, quis adipiscing urna leo nec orci. Sed nec nulla auctor odio aliquet consequat. Ut nec nulla in ante ullamcorper aliquam at sed dolor. Phasellus fermentum magna in augue gravida cursus. Cras sed pretium lorem. Pellentesque eget ornare odio. Proin accumsan, massa viverra cursus pharetra, ipsum nisi lobortis velit, a malesuada dolor lorem eu neque.

Phasellus fermentum magna in augue gravida cursus. Cras sed pretium lorem. Pellentesque eget ornare odio. Proin accumsan, massa viverra cursus pharetra, ipsum nisi lobortis velit, a malesuada dolor lorem eu neque.

\subsection{Tendencias actuales en el estudio de las redes mutualistas}
Morbi rutrum odio eget arcu adipiscing sodales. Aenean et purus a est pulvinar pellentesque. Cras in elit neque, quis varius elit. Phasellus fringilla, nibh eu tempus venenatis, dolor elit posuere quam, quis adipiscing urna leo nec orci. Sed nec nulla auctor odio aliquet consequat. Ut nec nulla in ante ullamcorper aliquam at sed dolor. Phasellus fermentum magna in augue gravida cursus. Cras sed pretium lorem. Pellentesque eget ornare odio. Proin accumsan, massa viverra cursus pharetra, ipsum nisi lobortis velit, a malesuada dolor lorem eu neque.

Phasellus fermentum magna in augue gravida cursus. Cras sed pretium lorem. Pellentesque eget ornare odio. Proin accumsan, massa viverra cursus pharetra, ipsum nisi lobortis velit, a malesuada dolor lorem eu neque.

%----------------------------------------------------------------------------------------
%	ESTRUCTURA DE LA TESIS
%----------------------------------------------------------------------------------------

\section{Estructura de la tesis}
Morbi rutrum odio eget arcu adipiscing sodales. Aenean et purus a est pulvinar pellentesque. Cras in elit neque, quis varius elit. Phasellus fringilla, nibh eu tempus venenatis, dolor elit posuere quam, quis adipiscing urna leo nec orci. Sed nec nulla auctor odio aliquet consequat. Ut nec nulla in ante ullamcorper aliquam at sed dolor. Phasellus fermentum magna in augue gravida cursus. Cras sed pretium lorem. Pellentesque eget ornare odio. Proin accumsan, massa viverra cursus pharetra, ipsum nisi lobortis velit, a malesuada dolor lorem eu neque.
