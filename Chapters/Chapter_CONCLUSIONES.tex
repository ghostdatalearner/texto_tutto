% Chapter Template

\chapter{Conclusiones de la tesis} % Main chapter title

\label{chapterCONCLUSIONES} % Change X to a consecutive number; for referencing this chapter elsewhere, use \ref{ChapterX}

El objetivo de esta tesis ha sido contribuir al estudio del mutualismo en ecología mediante su modelado como redes cooperativas. La investigación ha seguido dos líneas de trabajo, el modelado dinámico y la descripción estructural de este tipo de comunidades. En ambas se han propuesto innovaciones teóricas que se han puesto a prueba con una de las colecciones más amplias y fiables de datos de redes mutualistas.

Estas son las principales conclusiones que se han alcanzado:

\begin{enumerate}
\item Se han propuesto dos modelos de dinámica mutualista derivados de la ecuación logística. Ambos solucionan los problemas de estabilidad del modelo de May y la paradoja de Levins, derivada de la fórmula de Pearl, y permiten un tratamiento analítico más simple que los llamados de \textit{tipo II}.
	\begin{enumerate}
	\item Se ha demostrado como una pequeña modificación permite resolver ambos problemas. El primer modelo funciona con \textit{capacidad de carga constante}, con independencia de la abundancia de individuos de las especies mutualistas. Este modelo permite describir la dinámica habitual del mutualismo, sus puntos fijos y el \textit{saddle point} que marca la supervivencia de la comunidad. Se puede resolver de forma analítica y extender del caso simple de dos especies al más general. Las simulaciones numéricas han permitido reproducir lo que preveía el análisis.
	\item Se ha demostrado como extender la idea original del coeficiente de fricción $\alpha$ constante de la ecuación logística original para incluir los efectos del mutualismo, sin tener que recurrir a un tercer término. En el segundo modelo la fricción no es constante sino que crece con el número de individuos de las especies benefactoras. Esto conduce a que las especies alcancen puntos de equilibrio máximos que dependen de la abundancia de individuos de las otras especies, tal y como ocurre en la realidad. Por ello lo hemos denominado \textit{modelo con saturación del beneficio}. 
	El análisis de estabilidad de este modelo es más sencillo que para los modelos habituales de la literatura. Además, se ha explicado como se puede encontrar la divisoria que separa las cuencas de extinción y supervivencia, que es una ley de potencia para el caso de dos especies.
	\item Se ha comprobado la eficiencia computacional de la simulación estocástica a la hora de integrar las ecuaciones. Este tipo de simulación permite introducir de manera simple perturbaciones externas que suceden de manera habitual en la naturaleza. Los experimentos numéricos con este modelo y unas redes muy simples han mostrado la gran riqueza y complejidad de la dinámica del mutualismo. Para ello se ha desarrollado un software \textit{ad hoc} llamado \texttt{SIGMUND}, que se ha publicado en modalidad \texttt{Open Source} y que cualquier investigador puede descargar desde \texttt{github} clonando el repositorio \url{https://github.com/jgalgarra/sigmund}.
	\end{enumerate}
	

\item Se ha utilizado la \textit{descomposición k-core} como herramienta de análisis estructural del mutualismo en ecología. 
	\begin{enumerate}
	\item Se ha demostrado como las \textit{k-magnitudes} definidas como propiedades basadas en la topología de la red, permiten describir propiedades locales y globales del mutualismo. El $\overline {k}_{radius}$ actúa como medida de la compacidad de la red y guarda una alta correlación con la medida de anidamiento $NODF$. El ${k}_{degree}$ permite refinar la ordenación que proporciona el grado, al crear una escala en la que hay una probabilidad muy inferior de repetición de valores. Además, muestra una alta correlación con el observable $Modularity$.
	\item Uusando el análisis de modelo nulo se ha comprobado que $\overline {k}_{radius}$ es una medida de compacidad, que es una propiedad que se presenta solo en un subconjunto de redes significativamente anidadas.
	\item Mediante el experimento de recableado hemos comprobado como el anidamiento y la compacidad desaparecen de forma casi simultánea en las redes binarias que parten de un estado muy ordenado. Este comportamiento es menos habitual en redes pequeñas o con una fuerte asimetría de especies.
	\end{enumerate}

\item La visualización de datos es una gran ayuda para la investigación, porque permite observar detalles estructurales. Para ello, las herramientas gráficas deben estar adaptadas a las propiedades de la información. Los gráficos más usados en el estudio de las comunidades mutualistas se vuelven muy confusos cuando la red tiene unas pocas decenas de especies. 
	\begin{enumerate}
	\item Se ha mostrado como el \textit{diagrama polar} utiliza la descomposición como mecanismo de reducción de la información. Permite percibir en qué grado la red es jerárquica y es útil para comparar redes con independencia de sus tamaños.
	\item Se ha demostrado como el \textit{diagrama zigurat} representan todas las especies y enlaces y revela con claridad la estructura de \textit{k-shells} de la red. Tiene aplicación para comparar la evolución temporal de una red, ya sea por extinciones parciales, por experimentos de reconfiguración o por cualquier otra circunstancia que altere el número de especies y su conectividad. Los diagramas de zigurat tienden a adoptar una serie de figuras tipo que permiten deducir a simple vista algunas propiedades de la red.
	\item Se ha liberado en modo \texttt{Open Source} el software que lleva a cabo el análisis de las \textit{k-magnitudes} y la construcción de los diagramas. Es el paquete \texttt{R} denominado \texttt{kcorebip} que puede instalarse con el comando \texttt{install\_github}$("$\path{jgalgarra/kcorebip}$")$.
	\end{enumerate}

\item La resistencia de las redes mutualistas es una cuestión de gran importancia para la conservación de la biovidersidad. La definición del término resistencia no es neutral, hemos comprobado como conduce a resultados aparentemente contradictorios.
	\begin{enumerate}
	\item Hemos demostrado que el índice ${k}_{risk}$, basado en la \textit{descomposición k-core} es el mejor para identificar qué especies pueden producir una degradación más rápida de la componente gigante de la red, cuando se trabaja con el supuesto de que las extinciones primarias pueden proceder de ambas clases. Un pequeño porcentaje de extinciones, en torno al $12\%$ para redes de más de $100$ especies, reduce el tamaño de la componente gigante a menos de la mitad.
	\item Hemos demostrado que en el proceso clásico de destrucción de la red, provocando extinciones primarias solo en una de las clases, el resultado varía de manera muy significativa en función de la magnitud que se mida. El índice $MusRank$ es el más eficaz para identifcar las especies animales clave cuando se mide el porcentaje de especies vegetales supervivientes. Por el contrario, ${k}_{degree}$ es más conveniente si lo que se desea es mantener la componente gigante. La variedad de índices es útil para el establecimiento de políticas de conservación. Dependiendo de los objetivos fijados los responsables podrán elegir el más adecuado.
	\end{enumerate}
	
\end{enumerate}